%
% this file is encoded in utf-8% v2.0 (Apr. 5, 2009)
%
% !Tex program = xelatex
%\RequirePackage[2020-02-02]{latexrelease}
\documentclass[12pt, a4paper]{definitions/stylesheet}
\usepackage{definitions/definitions}

\usepackage[verbose,tmargin=2.5cm,bmargin=2.5cm,lmargin=3cm,rmargin=2cm, 
footskip=1.5cm,a4paper]{geometry}
\usepackage{times}
\usepackage{verbatim}
\usepackage{color}
\usepackage{url}
\usepackage{graphicx}
\usepackage{array}
\usepackage{wallpaper}
\usepackage[hypertexnames=false, colorlinks, linkcolor=black, citecolor=black, 
urlcolor=black, unicode]{hyperref}
\usepackage[printwatermark]{xwatermark}
\usepackage{graphicx}
\usepackage{tikz}
\usepackage{xCJKnumb}
\usepackage{indentfirst}
\usepackage{caption}
\usepackage{subcaption}
% algorithm 
\usepackage{algorithm} 
\usepackage{algorithmic}
\usepackage[nospace]{cite}  % for smart citation
\usepackage{multirow}
\usepackage{listings}
\usepackage{amsmath}
\usepackage{tabu}

\renewcommand{\algorithmicrequire}{ \textbf{Input:}} % use Input not require 
\renewcommand{\algorithmicensure}{ \textbf{Output:}} % use Output not ensure

% Set the default fonts
\setmainfont{Times New Roman}
\setCJKmainfont[AutoFakeBold=true,AutoFakeSlant=true]{標楷體}


\begin{document}

% global CJK setting
\setlength{\parindent}{2em}  %%% ZZZ %%%  段首內縮兩格

% 載入定義
%
% this file is encoded in utf-8
% v2.0 (Apr. 5, 2009)

% 單位
\renewcommand{\CollegeZh}{理工}
\renewcommand{\CollegeEn}{Science and Engineering}
\renewcommand{\DepartmentZh}{資訊工程}
\renewcommand{\DepartmentEn}{Computer Science and Information Engineering}
\renewcommand{\DepartmentLevel}{碩士班}

% 論文
\renewcommand{\ThesisType}{碩士論文}
\renewcommand{\Draft}{}
\renewcommand{\TitleEn}{English Title}
\renewcommand{\TitleZh}{中文題目}
\renewcommand{\KeywordsZh}{關鍵字1,關鍵字2,關鍵字3,關鍵字4}
\renewcommand{\KeywordsEn}{Abstract1, Abstract2, Abstract3, Abstract4}

% 提交日期
\renewcommand{\Semester}{一百一十一}		%學年度
\renewcommand{\SemesterTerm}{二}			%學期
\renewcommand{\SubmissionYearZh}{111}	%中華民國紀年年份
\renewcommand{\SubmissionMonthZh}{7}		%中文月份
\renewcommand{\SubmissionYearEn}{2022}		%公元紀年
\renewcommand{\SubmissionMonthEn}{July}		%英文月份

% 研究生與指導教授
\renewcommand{\AuthorZh}{你的名字}				%作者中文姓名
\renewcommand{\AuthorEn}{Your name}		%作者英文姓名
\renewcommand{\AuthorId}{M11111111}				%作者學號
\renewcommand{\Laboratory}{資料系統實驗室}		%實驗室
\renewcommand{\AdvisorAZh}{林朝興}				%指導教授A中文姓名
\renewcommand{\AdvisorAEn}{Chow-Sing Lin}		%指導教授A英文姓名
\renewcommand{\AdvisorATitleZh}{教授}			%指導教授A職稱
\renewcommand{\AdvisorATitleEn}{Dr.}			%指導教授A職稱
\renewcommand{\AdvisorBZh}{}			     	%指導教授B中文姓名
\renewcommand{\AdvisorBEn}{}		            %指導教授B英文姓名
\renewcommand{\AdvisorBTitleZh}{}				%指導教授B職稱
\renewcommand{\AdvisorBTitleEn}{}				%指導教授B職稱


% 其他頁面資訊
\renewcommand{\PdfTitle}{國立臺南大學數位論文典藏.pdf}
\renewcommand{\PdfKeywords}{\KeywordsZh{, }\KeywordsEn}
%\renewcommand{\AbstractZhTitle}{論文摘要}
%\renewcommand{\AbstractEnTitle}{ABSTRACT}
%\renewcommand{\AcknowledgementTitle}{誌謝}
\renewcommand{\AcknowledgementYear}{111}
\renewcommand{\AcknowledgementMonth}{7}
%\renewcommand{\CommitteeFormTitle}{審定書}
%\renewcommand{\Roc}{\makebox[5em][s]{中華民國}}
%\renewcommand{\Year}{\makebox[2em][c]{年}}
%\renewcommand{\Month}{\makebox[2em][c]{月}}
%\renewcommand{\Date}{\makebox[2em][c]{日}}

% 更新預設頁面資訊
%\renewcommand{\contentsname}{目錄}
%\renewcommand{\listfigurename}{圖目錄}
%\renewcommand{\listtablename}{表目錄}
%\renewcommand{\bibname}{參考文獻}
%\renewcommand{\prechaptername}{第} % 出現在每一章的開頭的「第 x 章」
%\renewcommand{\postchaptername}{章}
%\renewcommand{\tablename}{表} % 在文章中 table caption 會以「表 x」表示
%\renewcommand{\figurename}{圖} % 在文章中 figure caption 會以「圖 x」表示
%\renewcommand{\lstlistingname}{程式碼}

% 表格自動換行
\newcommand{\tabincell}[2]{\begin{tabular}{@{}#1@{}}#2\end{tabular}}
\hypersetup{
	pdftitle={\PdfTitle},
	pdfauthor={\SchoolZh{ }\DepartmentZh{學系}\DepartmentLevel{ }\AuthorZh},
	pdfsubject={\TitleZh},
	pdfkeywords={\PdfKeywords}}

% 如果不需要以中文數字一、二、三呈現章別,例如「第一章」
% 則請把下列介於 >>> 與 <<< 之間
% 的文字行關掉 (行首加上百分號), 會以「第 1 章」呈現
%% 中文數字章別 >>>
\input{content/chnum.tex}
%% <<< 中文數字章別

%%% 以下是載入前頁、本文、後頁
% 請勿更動
% 如需針對個別章節獨立編譯
% 請在 my_chapters.tex 檔裡對個別章節的 \input 指令以行首百分號方式做開關。

%\watermark
%\ptitlepage  % 論文書皮頁
%\includepdf[pages={1}]{content/frontpages/titlepage.pdf}

\includepdf[pages={1}]{content/frontpages/front.pdf}     %論文封面
\includepdf[pages={1}]{content/frontpages/bookname.pdf}   %書名頁

%\poralzh  % 論文口試委員會審定書
% \includepdf[pages={1}]{content/frontpages/oral.pdf}
%\poralen
% \includepdf[pages={1}]{content/frontpages/oral2.pdf}

\newpage
\setcounter{page}{1}
\pagenumbering{roman}

\begin{abstractZh}  %中文摘要
摘要摘要摘要摘要摘要摘要摘要摘要摘要摘要摘要摘要摘要摘要
\end{abstractZh}

\begin{abstractEn}  %英文摘要
Abstract Abstract Abstract Abstract Abstract Abstract Abstract Abstract
\end{abstractEn} 

\newpage
\begin{acknowledgement} %誌謝
\par
致謝致謝致謝致謝致謝致謝致謝致謝致謝致謝致謝致謝致謝致謝
\end{acknowledgement}

{\singlespacing\selectfont
	\newpage\phantomsection\addcontentsline{toc}{chapter}{\contentsname}
	\tableofcontents %目錄

	\newpage\phantomsection\addcontentsline{toc}{chapter}{\listtablename}
	\listoftables  %表目錄
	
	\newpage\phantomsection\addcontentsline{toc}{chapter}{\listfigurename}
	\listoffigures  %圖目錄
}


\newpage
\setcounter{page}{1}
\pagenumbering{arabic}

\watermark
\chapter{導論 (Introduction)}\label{chapter:introduction}
	導論導論導論\cite{Girshick2015}。\par

如圖\ref{fig1}所示,......\par
\begin{figure}[ht]
	\centering
	\includegraphics[scale=0.5]{"content/images/d3420288"}
	\caption{範例圖}
	\label{fig1}
\end{figure}\par


本論文的主要貢獻如下列幾點:
\begin{enumerate}
    \item  
    提出......
    \item 
    改善....
    \item
    改善....
\end{enumerate} 

本論文內容組織如下。第一章為...\par









\chapter{相關研究 (Related Work)}\label{chapter:related work}
	相關研究相關研究相關研究相關研究相關研究相關研究相關研究相關研究。\par
\chapter{架構架構架構 (Architecture Architecture Architecture Architecture)}\label{chapter:methods}
	\input{"content/chapter 3 methods"}			
\chapter{實驗結果 (Experimental Results)}\label{chapter:experimental results}
	
\section{評估指標}
如公式\ref{eq23}所示....。\par
\begin{equation}
	\label{eq23}
	c^2 = a^2 + b^2
\end{equation}\par

\newpage
\section{定量結果 (Quantitative Results)}
表\ref{tb1}為我們的實驗結果.....。\par

\begin{table}[!h]
	\centering
	\caption{實驗結果}
	\label{tb1}
	\begin{tabu}{l|[1.5pt]cccc|[1.5pt]c|[1.5pt]c}
		\hline
		Method  & Metric1 & Metric2 & Metric3 & Metric4 & Metric5 & Metric6 
		\\\tabucline[1.5pt]{-}
		Method1   & 1    & 2   & 3  & 4  & 5  & 6   \\ 
		Method2   & 7    & 8   & 9  & 10 & 11 & 12  \\
		Method3   & 13   & 14  & 15 & 16 & 17 & 18  \\ 
		\tabucline[1.5pt]{-}
		Method4   & 19   & 20  & 21 & 22 & 23 & 24  \\ 
		\hline
	\end{tabu}
\end{table}









\chapter{總結與未來研究 (Conclusion and Future Work)}\label{chapter:conclusion 
and future work}
	結論結論結論。\par

未來研究未來研究未來研究未來研究未來研究。\par





	
	
	
	
	

\input{content/backpages.tex}

% 附錄
\clearpage
\phantomsection\addcontentsline{toc}{chapter}{附錄一{ }附錄一附錄一}

\centerline{\Large{\textbf{附錄一\hs 附錄一附錄一}}}

\vspace{2\baselineskip}

%\renewcommand\arraystretch{0.5}
\begin{table}[htbp]
	\caption{線上手機位置識別混淆矩陣}
	\centering
	\begin{tabular}{|c|c|c|c|c|c|c|}
		\hline
		\multicolumn{2}{|c|}{}&\multicolumn{5}{c|}{Predicted Class} \\  
		\cline{3-7}
		\multicolumn{2}{|c|}{} & FrontPocket & BackPocket & CoatPocket & ShoulderBag & BackBag \\ \hline       
		\multirow{5}{*}{\tabincell{c}{Actual\\Class}}& 
		FrontPocket& 
		184&8&8&0&0\\ \cline{2-7}
		&BackPocket&
		10&187&3&0&0\\ \cline{2-7}
		&CoatPocket&
		11&0&183&2&4\\ \cline{2-7}                               
		&ShoulderBag& 
		1&0&10&158&31\\ \cline{2-7}
		&BackBag& 
		0&0&18&15&167\\ \hline                                                                
	\end{tabular}
\end{table}

%\input{content/appendix2.tex}
%\input{content/appendix3.tex}
%\input{content/appendix4.tex}
%\input{content/appendix5.tex}
%\input{content/appendix6.tex}
%\input{content/appendix7.tex}
%\input{content/appendix8.tex}


%\clearpage % to make sure all CJK characters are processed
%\end{CJK}  %%% ZZZ %%%
\end{document} 
 
